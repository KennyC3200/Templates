% Basics
\usepackage[utf8]{inputenc}
\usepackage[T1]{fontenc}
\usepackage{textcomp}
\usepackage[a4paper, left=1in, right=1in, top=1in, bottom=1in]{geometry}
% \renewcommand\familydefault{\sfdefault}

\usepackage[Rejne]{fncychap}

\usepackage{amsmath,amsfonts,amsthm,amssymb,mathtools}
\usepackage[varbb]{newpxmath}
\usepackage{xfrac}
\usepackage[usenames,dvipsnames]{xcolor} % usenames, dvipsnames adds more colours
\usepackage{hhline}
\usepackage{comment}
\usepackage{tasks}
\usepackage{enumerate} 
\usepackage{enumitem} 
\usepackage{titlesec}
\usepackage[most]{tcolorbox}
\usepackage{lipsum}
\usepackage{tabularx}
\usepackage[labelfont=bf]{caption}
\usepackage{subfig}
\usepackage{cancel} 
\usepackage{physics} 
\usepackage[bookmarks]{hyperref}
\usepackage{array}
\usepackage{float}
% \usepackage{standalone}
\usepackage{graphicx}
\usepackage{forest}

% Tables 
\numberwithin{table}{section}

% Inkscape Figures
\usepackage{import}
\usepackage{xifthen}
\usepackage{pdfpages}
\usepackage{transparent}
\newcommand{\incfig}[2][1]{
    \def\svgwidth{#1\columnwidth}
    \import{../figures/}{#2.pdf_tex}
}

\pdfsuppresswarningpagegroup=1

% Chemistry
\usepackage{lewis} 
\usepackage{bohr}
\usepackage[version=4]{mhchem}

% Page setup
\usepackage{fancyhdr}
\hypersetup{hidelinks}
\pagenumbering{arabic}
\pagestyle{fancy}
\fancyhf{} % clear all header and footer fields
% \fancyhead[R]{Unit \thechapter}
\newcommand{\myheader}[2]{\MakeUppercase{#1}: \MakeUppercase{#2}}
% \fancyhead[R]{\myheader{\leftmark}{\rightmark}}
\renewcommand{\chaptermark}[1]{\markboth{#1}{}}
\renewcommand{\sectionmark}[1]{\markright{#1}}
\fancyhead[R]{\myheader{\leftmark}{\rightmark}}
\fancyfoot[R]{\thepage}
\setlength{\parindent}{0pt}

% Show subsubsections
\setcounter{tocdepth}{3}
\setcounter{secnumdepth}{3}

% Required for the Grid
\usetikzlibrary{calc}

% Theorems
\usepackage{thmtools}
\usepackage[framemethod=TikZ]{mdframed}

% Note environment 
\declaretheoremstyle[
    headfont=\bfseries\sffamily\color{ProcessBlue!70!black}, bodyfont=\normalfont,
    headpunct= :,
    mdframed={
        linewidth=1pt,
        rightline=false, topline=false, bottomline=false,
        linecolor=ProcessBlue, backgroundcolor=ProcessBlue!5,
        innerbottommargin=10pt
    }
]{note}
\declaretheorem[style=note,name=Note,numbered=no]{note}

% Solution environment
\declaretheoremstyle[
    headfont=\bfseries\sffamily\color{NavyBlue!70!black}, 
    bodyfont=\normalfont,
    % headpunct=,
    mdframed={
        linewidth=1pt,
        rightline=false, topline=false, bottomline=false, linecolor=NavyBlue, innerbottommargin=5pt
    }
]{solution}
\declaretheorem[style=solution,name=Solution,numbered=no]{solution}

% Definition environment
\declaretheoremstyle[
    headfont=\bfseries\sffamily, 
    bodyfont=\normalfont, 
    mdframed={
        nobreak,
        linewidth=1pt,
        % innerbottommargin=10pt,
        rightline=false, topline=false, bottomline=false,
    }
]{definition}
\declaretheorem[style=definition,name=Definition,numberwithin=chapter]{definition}

% Example environment
\declaretheoremstyle[
    headfont=\bfseries\sffamily\color{Fuchsia!70!black}, 
    bodyfont=\normalfont,
    mdframed={
        linewidth=1pt,
        rightline=false, topline=false, bottomline=false,
        linecolor=Fuchsia, backgroundcolor=Fuchsia!5,
    }
]{exampleswap}
\declaretheorem[style=exampleswap,name=Example,numbered=no]{exampleswap}

% Example solution environment
\declaretheoremstyle[
    headfont=\bfseries\sffamily\color{Fuchsia!70!black}, 
    bodyfont=\normalfont,
    mdframed={
        linewidth=1pt,
        rightline=false, topline=false, bottomline=false,
        linecolor=Fuchsia,
    }
]{examplesolution}
\declaretheorem[style=examplesolution,name=Solution,numbered=no]{examplesolution}

\newenvironment{example}[1]
{
    \begin{exampleswap}
        #1
    \end{exampleswap}
    \vspace{-1.7em}
    \begin{examplesolution}
}
{\end{examplesolution}}

\newenvironment{2example}[1]
{
    \begin{exampleswap}[#1]
}
{
\end{exampleswap}}

% Remark environment
\declaretheoremstyle[
    headfont=\bfseries\sffamily,
    bodyfont=\normalfont
]{remark}
\declaretheorem[style=remark, name=Remark, numbered=no]{remark}

% Chapter -> Unit
\renewcommand{\thechapter}{\arabic{chapter}} % Changes chapter numbering to Arabic numerals
\renewcommand{\chaptername}{Unit} % Renames "Chapter" to "Lecture"
\newcommand{\unit}[1]{%
  \chapter{#1}
}
